%----------------------------------------------------------------------------------------
%	PACKAGES AND OTHER DOCUMENT CONFIGURATIONS
%----------------------------------------------------------------------------------------

\documentclass[a4paper,12pt]{article}
%\usepackage[english]{babel}
\usepackage{amsmath}
\usepackage{graphicx}
\usepackage[colorinlistoftodos]{todonotes}
\usepackage{fullpage}
\usepackage{multicol,multirow}
\usepackage{tabularx}
\usepackage{ulem}
\usepackage[utf8]{inputenc}
\usepackage[russian]{babel}
\usepackage{amsmath}
\usepackage{amssymb}
\usepackage{listings}
\usepackage{titlesec}

\usepackage{indentfirst}

\usepackage{longtable}
\usepackage{ltxtable}
\usepackage{listings}
\usepackage{color}

\definecolor{codegreen}{rgb}{0,0.6,0}
\definecolor{codegray}{rgb}{0.5,0.5,0.5}
\definecolor{codepurple}{rgb}{0.58,0,0.82}
\definecolor{backcolour}{rgb}{0.95,0.95,0.92}

\lstdefinestyle{mystyle}{
    %backgroundcolor=\color{backcolour},
    commentstyle=\color{codegreen},
    keywordstyle=\color{magenta},
    numberstyle=\tiny\color{codegray},
    stringstyle=\color{codepurple},
    basicstyle=\footnotesize,
    breakatwhitespace=false,
    breaklines=true,
    captionpos=b,
    keepspaces=true,
    numbers=left,
    numbersep=5pt,
    showspaces=false,
    showstringspaces=false,
    showtabs=false,
    tabsize=2
}
\lstset{style=mystyle}

\usepackage{hyperref}
\hypersetup{
    colorlinks=true,
    linkcolor=black,
    filecolor=magenta,
    urlcolor=blue,
}

\urlstyle{same}

%\usepackage{geometry}
%\geometry{top=2cm}
%\geometry{bottom=2cm}
%\geometry{left=1.5cm}
%\geometry{right=1.5cm}

\setlength{\parskip}{0.5cm}


\begin{document}

\begin{titlepage}

\newcommand{\HRule}{\rule{\linewidth}{0.5mm}} % Defines a new command for the horizontal lines, change thickness here

\center % Center everything on the page



%----------------------------------------------------------------------------------------
%	HEADING SECTIONS
%----------------------------------------------------------------------------------------
\textsc{\large Московский Авиационный Институт\\(национальный исследовательский университет)}\\[1.5cm] % Name of your university/college

%----------------------------------------------------------------------------------------
%	LOGO SECTION
%----------------------------------------------------------------------------------------
\includegraphics[width=0.25\textwidth]{mai.png}\\[1cm]
%----------------------------------------------------------------------------------------
\vspace{40px}

%----------------------------------------------------------------------------------------
%	TITLE SECTION
%----------------------------------------------------------------------------------------
\textsc{\Large Курсовая работа по курсу\\ \guillemotleft Численные методы\guillemotright}\\[0.5cm]

\HRule \\[0.4cm]
{ \huge \bfseries Метод Монте-Карло для вычисления кратных интегралов}\\[0.4cm] % Title of your document
\HRule \\[1.5cm]

%----------------------------------------------------------------------------------------
%	AUTHOR SECTION
%----------------------------------------------------------------------------------------
\begin{minipage}{0.4\textwidth}
\begin{flushleft} \large
\emph{Студент:}\\
Макаров Никита, 8о-306Б
\end{flushleft}
\end{minipage}
~
\begin{minipage}{0.4\textwidth}
\begin{flushright} \large
\emph{Руководитель:} \\
Ревизников Д.Л.
\end{flushright}
\end{minipage}\\[2cm]

%----------------------------------------------------------------------------------------
%	DATE SECTION
%----------------------------------------------------------------------------------------

\vspace{120px}
{\large Москва\\2015}\\[2cm] % Date, change the \today to a set date if you want to be precise

\vfill % Fill the rest of the page with whitespace

\end{titlepage}



%----------------------------------------------------------------------------------------
%	СОДЕРЖАНИЕ
%----------------------------------------------------------------------------------------
\tableofcontents

%----------------------------------------------------------------------------------------
%	ВВЕДЕНИЕ
%----------------------------------------------------------------------------------------
\newpage
\section{Введение}

Методами Монте-Карло называют численные методы решения математических задач при помощи моделирования случайных величин. Однако, решать методами Монте-Карло можно любые математические задачи, а не только задачи вероятностного происхождения, связанные со случайными величинами.

Важнейшим приемом построения методов Монте-Карло является сведение задачи к расчету математических ожиданий. Так как математические ожидания чаще всего представляют собой обычные интегралы, то центральное положение в теории метода Монте-Карло занимают методы вычисления интегралов.

Преимущества недетерминированных методов особенно ярко проявляются при решении задач большой размерности, когда применение традиционных детерминированных методов затруднено или совсем невозможно.

До появления ЭВМ методы Монте-Карло не могли стать универсальными численными методами, ибо моделирование случайных величин вручную - весьма трудоемкий процесс. Развитию методов Монте-Карло способствовало бурное развитие ЭВМ. Алгоритмы Монте-Карло сравнительно легко программируются и позволяют производить расчеты во многих задачах, недоступных для классических численных методов. Так как совершенствование ЭВМ продолжается, есть все основания ожидать дальнейшего развития методов Монте-Карло и дальнейшего расширения области их применения.


%----------------------------------------------------------------------------------------
%	СУЩНОСТЬ
%----------------------------------------------------------------------------------------
\newpage
\section{Идея метода Монте-Карло}

Важнейший прием построения методов Монте-Карло~--- сведение задачи к расчету математических ожиданий. Пусть требуется найти значение $m$ некоторой изучаемой величины. С этой целью выбирают такую случайную величину $X$, математическое ожидание которой равно $m : M[X] = m$. Практически же поступают так: вычисляют $N$ возможных значений $x_i$ случайной величины $X$ и находят их среднее арифметическое
\[
	\bar{x} = \frac{1}{N} \sum_{i=1}^N x_i.
\]

Так как последовательность одинаково распределенных случайных величин, у которых существуют математические ожидания, подчиняется закону больших чисел, то при $N \to \infty $ среднее арифметическое этих величин сходится по вероятности к математическому ожиданию. Таким образом, при больших $N$ величина $\bar{x} \approx m $.

В методе Монте-Карло данные вырабатываются искусственно путем использования некоторого генератора случайных чисел в сочетании с функцией распределения вероятностей для исследуемого процесса.



%----------------------------------------------------------------------------------------
%	ПОГРЕШНОСТЬ
%----------------------------------------------------------------------------------------
\newpage
\section{Погрешность метода Монте-Карло}

Для оценки величины $m$ смоделируем случайную величину $X$ с математическим ожиданием $M[X]=m$. Выберем $N$ независимых реализаций $x_1,...,x_N$ случайной величины $X$ и вычислим среднее арифметическое:
\[
	\bar{x} = \frac{1}{N} \sum_{i=1}^N x_i.
\]

Также предположим, что случайная величина $X$ имеет конечную дисперсию:
\[
	D[X] = M[X^2] - (M[X])^2.
\]

Из центральной предельной теоремы имеем:
\begin{equation}
	P \left \{ |\bar{X} - m| < t_\beta \sqrt{\frac{D[X]}{N}} \right \} \approx 2\Phi(t_\beta) = \beta.
\end{equation}

Из уравнения $(1)$ получаем верхнюю границу ошибки с коэффициентом доверия $\beta$:
\begin{equation}
	\varepsilon = t_\beta \sqrt{\frac{D[X]}{N}}.
\end{equation}

Задав некоторое значение $\varepsilon$ и $\beta$ при известном $\sigma = \sqrt{D[X]}$ можно определить необходимое количество испытаний, обеспечивающее точность $\varepsilon$ с надежностью $\beta$:
\begin{equation}
	N = \frac{\sigma^2 \cdot t_\beta^2}{\varepsilon^2}.
\end{equation}

Обычно при решении реальных задач значение дисперсии неизвестно, а следовательно неизвестен параметр $\sigma$. Чтобы определить приближенное значение $\sigma$ проводят некоторое начальное количество испытаний $N_0$. Затем по результатам этих испытаний определяется приближенное значение дисперсии:
\begin{equation}
	D[X] = \frac{1}{N_0} \sum_{i=1}^{N_0} x_i^2 - \left ( \frac{1}{N_0} \sum_{i=1}^{N_0} x_i \right ) ^2.
\end{equation}

Зная значение $D[X]$ можем получить приближенное значение $N$:
\begin{equation}
	N = \frac{(D[X])^2 \cdot t_\beta^2}{\varepsilon^2}.
\end{equation}



%----------------------------------------------------------------------------------------
%	УСРЕДНЕНИЕ 
%----------------------------------------------------------------------------------------
\newpage
\section{Вычисление интегралов усреднением подынтегральной функции}

Пусть требуется вычислить интеграл
\begin{equation}
	I = \int\limits_a^b \varphi(x)dx.
\end{equation}

Предположим, что имеется случайная величина $X$, равномерно распределенная на интервале $(a, b)$ с плотностью вероятности $f(x) = \frac{1}{b-a}$. Тогда имеем математическое ожидание:
\begin{equation}
	M[\varphi(x)] = \int\limits_a^b \varphi(x)f(x)dx = \frac{1}{b-a}\int\limits_a^b \varphi(x)dx.
\end{equation}

Из уравнения $(4)$ следует:
\begin{equation}
	\int\limits_a^b \varphi(x)dx = (b-a)M[\varphi(x)].
\end{equation}

Если теперь заменить математическое ожидание $M[\varphi(x)]$ его оценкой~--- выборочным средним, то получим оценку интеграла $(3)$:
\begin{equation}
	I = (b-a)\frac{{\displaystyle \sum_{i=1}^N \varphi(x_i)}}{N},
\end{equation}
где $x_i$~--- значения случайной величины $X$, а $N$~--- количество испытаний. Так как $X \sim R(a,b)$, то очевидно, что $x_i = a+(b-a)\eta$, где $\eta$~--- случайное число из интервала $(0,1)$.



%----------------------------------------------------------------------------------------
%	ПО 
%----------------------------------------------------------------------------------------
\newpage
\section{Описание программы}

В ходе выполнения данной работы была разработанна программа на языке {\ttfamily MATLAB}, вычисляющая кратные интегралы произвольной размерности.

{\textbf{Функции}}
\begin{itemize}
	\item {\ttfamily ndIntegral(f,a,b,G,N,t)}~--- вычисляет значение интеграла от функции $f$ по области $G$ с ограниченями области интегрирования $a$ и $b$ за $N$ испытаний при коэффициенте доверия $t$;
	\item {\ttfamily randInRange(a,b)}~--- возвращает случайный вектор в интервале $(a,b)$;
	\item {\ttfamily checkPoint(x,G)}~--- проверяет принадлежность точки $x$ области $G$;
	\item {\ttfamily test1d(N,error,t)}~--- тест 1-мерного интеграла с начальным количеством испытаний {\ttfamily N} и погрешностью {\ttfamily error} при коэффициенте доверия {\ttfamily t};
	\item {\ttfamily test2d(N,error,t)}~--- тест 2-мерного интеграла с начальным количеством испытаний {\ttfamily N} и погрешностью {\ttfamily error} при коэффициенте доверия {\ttfamily t};
	\item {\ttfamily test3d(N,error,t)}~--- тест 3-мерного интеграла с начальным количеством испытаний {\ttfamily N} и погрешностью {\ttfamily error} при коэффициенте доверия {\ttfamily t};
	\item {\ttfamily test6d(N,error,t)}~--- тест 6-мерного интеграла с начальным количеством испытаний {\ttfamily N} и погрешностью {\ttfamily error} при коэффициенте доверия {\ttfamily t};
\end{itemize}



%----------------------------------------------------------------------------------------
%	ТЕСТИРОВАНИЕ 
%----------------------------------------------------------------------------------------
\newpage
\section{Тестирование разработанного ПО}

\subsection{1-мерный интеграл}

Вычислим интеграл, не имеющий первообразной в классе элементарных функций:
\begin{equation}
	\frac{1}{\sqrt{2\pi}} \int\limits_{0}^{3} e^{-\frac{t^2}{2}}\:dt.
\end{equation}

Его значение известно и соответствует $\Phi(3)$ в таблице значений функции Лапласа, то есть $0.49865$.

Вычислим значение с помощью разработанного ПО.

\begin{lstlisting}[language=MATLAB]
>> MonteCarlo.test1d(50,0.05,3);
Error of integration:
    0.1792

First approximation of integral value:
    0.5087

Minimal necessary N:
    643

Error of integration:
    0.0486

Integral value:
    0.4992
\end{lstlisting}

Разность с точным решением составила $0.0005$.


\newpage
\subsection{2-мерный интеграл}

Вычислим интеграл
\begin{equation}
	I = \iint\limits_G (x+y)\:dx\:dy,
\end{equation}
\[
	G: 0 \leq x \leq 2,\:
	 x^2 \leq y \leq 2x.
\]

Для него известно аналитическое решение: $I = 3.4(6)$.

Вычислим значение с помощью разработанного ПО.

\begin{lstlisting}[language=MATLAB]
>> MonteCarlo.test2d(100,0.1,3);
Error of integration:
    3.9595

First approximation of integral value:
    3.5285

Minimal necessary N:
    156781

Error of integration:
    0.0922

Integral value:
    3.4582
\end{lstlisting}

Разность с точным решением составила $0.008$.


\newpage
\subsection{3-мерный интеграл}

Вычислим интеграл
\begin{equation}
	I = \iiint\limits_G 10x\:dx\:dy\:dz,
\end{equation}
\[
	G: 0 \leq x \leq 1,\:
	   0 \leq y \leq \sqrt{1-x^2},\:
	   0 \leq z \leq \sqrt{\frac{x^2 + y^2}{2}}.
\]

Для него известно аналитическое решение: $I = 1$.

Вычислим значение с помощью разработанного ПО.

\begin{lstlisting}[language=MATLAB]
>> MonteCarlo.test3d(1000,0.1,3);
Error of integration:
    0.2990

First approximation of integral value:
    0.9655

Minimal necessary N:
    943

Error of integration:
    0.1025

Integral value:
    1.0056
\end{lstlisting}

Разность с точным решением составила $0.0056$.


\newpage
\subsection{6-мерный интеграл}

Решим задачу о вычислении силы притяжения Земли и Луны. Искомое значение силы
\begin{equation}
	F = \sqrt{F_x^2 + F_y^2 + F_z^2},
\end{equation}
где
\[
	F_x = G \iint \iint\limits_{D \times D'} \iint \frac{\rho(x,y,z)\rho'(x',y',z')}{r^2}(x-x')\:dx\:dy\:dz\:dx'\:dy'\:dz',
\]
\[
	F_y = G \iint \iint\limits_{D \times D'} \iint \frac{\rho(x,y,z)\rho'(x',y',z')}{r^2}(y-y')\:dx\:dy\:dz\:dx'\:dy'\:dz',
\]
\[
	F_z = G \iint \iint\limits_{D \times D'} \iint \frac{\rho(x,y,z)\rho'(x',y',z')}{r^2}(z-z')\:dx\:dy\:dz\:dx'\:dy'\:dz',
\]
\[	
	r = \sqrt{(x-x')^2 + (y-y')^2 + (z-z')^2},
\]
Здесь под $D$ и $D'$ подразумеваются области интегрирования, то есть Земля и Луна.

Точное значение можно получить, если принять Землю и Луну за материальные точки ввиду расстояния между ними. Тогда искомая величина выражается формулой
\begin{equation}
	F = G \frac{m_1 \cdot m_2}{r^2}.
\end{equation}
и равна $1.98997 \cdot 10^{20}$ Н.

Вычислим значение с помощью разработанного ПО.

\begin{lstlisting}[language=MATLAB]
>>MonteCarlo.test6d(1000000,0.11e+20,3)
Integral value:
	1.9731e+20

Difference with correct answer:
	1.6823e+18
\end{lstlisting}

Разность с точным решением составила $1.6823 \cdot 10^{18}$, что очень неплохо, учитывая масштабы задачи.



%----------------------------------------------------------------------------------------
%	КОД 
%----------------------------------------------------------------------------------------
\newpage
\section{Исходный код программы}

\begin{lstlisting}[language=MATLAB]
classdef MonteCarlo
    % Integration by Monte Carlo method

    methods(Static)
        
        %% randInRange: Returns random value in range [a,b].
        function [x] = randInRange(a,b)
            n = length(a);
            x = zeros(1,n);
            for i = 1:n
                x(i) = a(i) + (b(i) - a(i)) .* rand();
            end
        end
        
        %% checkPoint: Checks, is point x in G area, or not.
        function [inArea] = checkPoint(x,G)
            n = length(G); % old version: n = length(x) !!!
            inArea = 1;
            for i = 1:n
                inArea = inArea && G{i}(x);
            end
        end
        
        %% ndIntegral: Computing n-dimensional definite integral 
         % at G area which no more than 
         % n-dimensional parallelepiped with properties a and b.
        function [I,c] = ndIntegral(f,a,b,G,N,t_beta)
            %t_beta = 3; % beta = 0.997
            fSum = 0; % sum of computed values f(x)
            fSumSquared = 0; % squared sum of f(x)
            n = 0; % amount of points found in G
            
            % generating N random vectors x
            for i = 1:N
                x = MonteCarlo.randInRange(a,b);
                % check conditions, x must be in [a,b]
                inArea = MonteCarlo.checkPoint(x,G); % bool value
                % adding f(x)
                if (inArea)
                    fSum = fSum + f(x);
                    fSumSquared = fSumSquared + f(x)^2;
                    n = n + 1;
                end
            end
            c = inf;
            
            % computing n-dimensional volume of figure
            V = prod(b-a);
            % computing integral value
            I = V * fSum / N;
            
            if (nargin == 6)
                fAvg = fSum / n;
                fSquaredAvg = fSumSquared / n;
                Omega = n / N;
                % computing standard deviation
                S1 = sqrt(fSquaredAvg - fAvg^2);
                S2 = sqrt(Omega * (1 - Omega));
                % computing error
                error = V*t_beta*(Omega*S1/sqrt(n) + abs(fAvg)*S2/sqrt(N));
                c = V*t_beta*(sqrt(Omega)*S1 + fAvg*S2);
                disp('Error of integration:');
                disp(error);
            end
        end
        
        %% TESTS
        
        %% test1d: computing 1-dimensional definite integral
        function [I] = test1d(N0, maxError, t_beta)
            f = @(x) 1/sqrt(2*pi) * exp(-(x^2)/2);

            % conditions of G area
            x1Cond = @(x) (0<=x(1) && x(1)<=3);
            G = {x1Cond};
            
            % limitations for every element in x
            a(1) = 0; b(1) = 3;
            
            [I,c] = MonteCarlo.ndIntegral(f,a,b,G,N0,t_beta);
            disp('First approximation of integral value:');
            disp(I);
            
            % computing min necessary N
            N = ceil((c/maxError)^2);
            
            if (N > N0)
                disp('Minimal necessary N:');
                disp(N);

                % computing more correct integral value
                [I] = MonteCarlo.ndIntegral(f,a,b,G,N,t_beta);
                disp('Integral value:');
                disp(I);
            end
        end
        
        %% test2d: computing 2-dimensional definite integral
        function [I] = test2d(N0, maxError, t_beta)
            f = @(x) x(1) + x(2);
            
            % conditions of G area
            x1Cond = @(x) (0<=x(1) && x(1)<=2);
            x2Cond = @(x) (x(1)^2<=x(2) && x(2)<=2*x(1));
            G = {x1Cond,x2Cond};
            
            % limitations for every element in x
            a(1) = 0; b(1) = 2;
            a(2) = 0; b(2) = 4;
            
            [I,c] = MonteCarlo.ndIntegral(f,a,b,G,N0,t_beta);
            disp('First approximation of integral value:');
            disp(I);
            
            % computing min necessary N
            N = ceil((c/maxError)^2);
            
            if (N > N0)
                disp('Minimal necessary N:');
                disp(N);

                % computing more correct integral value
                [I] = MonteCarlo.ndIntegral(f,a,b,G,N,t_beta);
                disp('Integral value:');
                disp(I);
            end
        end
        
        %% test3d: computing 3-dimensional definite integral
        function [I] = test3d(N0, maxError, t_beta)
            f = @(x) 10*x(1);
            
            % conditions of G area
            x1Cond = @(x) (0<=x(1) && x(1)<=1);
            x2Cond = @(x) (0<=x(2) && x(2)<=sqrt(1-x(1)^2));
            x3Cond = @(x) (0<=x(3) && x(3)<=((x(1)^2+x(2)^2)/2));
            G = {x1Cond,x2Cond,x3Cond};
            
            % limitations for every element in x
            a(1) = 0; b(1) = 1;
            a(2) = 0; b(2) = 1;
            a(3) = 0; b(3) = 1;
            
            [I,c] = MonteCarlo.ndIntegral(f,a,b,G,N0,t_beta);
            disp('First approximation of integral value:');
            disp(I);
            
            % computing min necessary N
            N = ceil((c/maxError)^2);
            
            if (N > N0)
                disp('Minimal necessary N:');
                disp(N);

                % computing more correct integral value
                [I] = MonteCarlo.ndIntegral(f,a,b,G,N,t_beta);
                disp('Integral value:');
                disp(I);
            end
        end

        %% test6d: computing 6-dimensional definite integral
        function [I] = test6d(N0, maxError, t_beta)
            % Solving the problem of the 
            % mutual attraction of two material bodies.
            
            gravityConst = 6.67e-11; % gravitational constant
            m1 = 6e+24; % mass of the Earth
            m2 = 7.35e+22; % mass of the Moon
            r = 384467000; % distance between Earth and Moon
            p1 = 5520; % avg density of Earth
            p2 = 3346; % avg density of Moon
            R1 = 6367000; % Earth radius
            R2 = 1737000; % Moon radius
            
            dist = @(x) sqrt((x(1)-(r+x(4)))^2 + ...
                             (x(2)-x(5))^2 + ...
                             (x(3)-x(6))^2);
            
            fx = @(x) (x(1)-(r+x(4))) / ((dist(x))^3);
            fy = @(x) (x(2)-x(5)) / ((dist(x))^3);
            fz = @(x) (x(3)-x(6)) / ((dist(x))^3);
            
            % conditions of G area
            x1Cond = @(x) (x(1)^2+x(2)^2+x(3)^2<=R1^2);
            x2Cond = @(x) (x(4)^2+x(5)^2+x(6)^2<=R2^2);
            G = {x1Cond,x2Cond};
            
            % limitations for every element in x
            a(1) = -R1; b(1) = R1;
            a(2) = -R1; b(2) = R1;
            a(3) = -R1; b(3) = R1;
            a(4) = -R2; b(4) = R2;
            a(5) = -R2; b(5) = R2;
            a(6) = -R2; b(6) = R2;
            
            [FxI,c] = MonteCarlo.ndIntegral(fx,a,b,G,N0,t_beta);
            Fx = gravityConst * p1 * p2 * FxI;
            
            [FyI,c] = MonteCarlo.ndIntegral(fy,a,b,G,N0,t_beta);
            Fy = gravityConst * p1 * p2 * FyI;
            
            [FzI,c] = MonteCarlo.ndIntegral(fz,a,b,G,N0,t_beta);
            Fz = gravityConst * p1 * p2 * FzI;
            
            I = sqrt(Fx^2 + Fy^2 + Fz^2);
            disp('Integral value:');
            disp(I);
                        
            F_correct = gravityConst * m1 * m2 / r^2;
            
            diff = abs(I - F_correct);
            disp('Difference with correct answer:');
            disp(diff);
        end 
    end
end
\end{lstlisting}

%----------------------------------------------------------------------------------------
%	ЛИТЕРАТУРА 
%----------------------------------------------------------------------------------------
\newpage
\section{Список литературы}

\begin{itemize}
	\item Кетков Ю.Л., Кетков А.Ю., Шульц М.М. MATLAB 7, программирование, числен- ные методы. - СПб.:БХВ-Петербург, 2005г. 752 стр. с илл.
	\item Г.М. Фихтенгольц Курс дифференциального и интегрального исчисления. Т. III, М.: Наука, 1956.- 656 с. с илл.
	\item Соболь И.М. Численные методы Монте-Карло, М. ФИЗМАТЛИТ, 1973.-312 с.
	\item Ермаков С.М. Метод Монте-Карло и смежные вопросы, М.: ФИЗМАТЛИТ, 1975. - 2-е изд.
\end{itemize}


\end{document}