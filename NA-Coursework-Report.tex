%----------------------------------------------------------------------------------------
%	PACKAGES AND OTHER DOCUMENT CONFIGURATIONS
%----------------------------------------------------------------------------------------

\documentclass[a4paper,12pt]{article}
%\usepackage[english]{babel}
\usepackage{amsmath}
\usepackage{graphicx}
\usepackage[colorinlistoftodos]{todonotes}
\usepackage{fullpage}
\usepackage{multicol,multirow}
\usepackage{tabularx}
\usepackage{ulem}
\usepackage[utf8]{inputenc}
\usepackage[russian]{babel}
\usepackage{amsmath}
\usepackage{amssymb}
\usepackage{listings}
\usepackage{titlesec}

\usepackage{longtable}
\usepackage{ltxtable}
\usepackage{listings}
\usepackage{color}

\definecolor{codegreen}{rgb}{0,0.6,0}
\definecolor{codegray}{rgb}{0.5,0.5,0.5}
\definecolor{codepurple}{rgb}{0.58,0,0.82}
\definecolor{backcolour}{rgb}{0.95,0.95,0.92}

\lstdefinestyle{mystyle}{
    %backgroundcolor=\color{backcolour},
    commentstyle=\color{codegreen},
    keywordstyle=\color{magenta},
    numberstyle=\tiny\color{codegray},
    stringstyle=\color{codepurple},
    basicstyle=\footnotesize,
    breakatwhitespace=false,
    breaklines=true,
    captionpos=b,
    keepspaces=true,
    numbers=left,
    numbersep=5pt,
    showspaces=false,
    showstringspaces=false,
    showtabs=false,
    tabsize=2
}
\lstset{style=mystyle}

\usepackage{hyperref}
\hypersetup{
    colorlinks=true,
    linkcolor=black,
    filecolor=magenta,
    urlcolor=blue,
}

\urlstyle{same}

%\usepackage{geometry}
%\geometry{top=2cm}
%\geometry{bottom=2cm}
%\geometry{left=1.5cm}
%\geometry{right=1.5cm}

\setlength{\parskip}{0.5cm}


\begin{document}

\begin{titlepage}

\newcommand{\HRule}{\rule{\linewidth}{0.5mm}} % Defines a new command for the horizontal lines, change thickness here

\center % Center everything on the page



%----------------------------------------------------------------------------------------
%	HEADING SECTIONS
%----------------------------------------------------------------------------------------
\textsc{\large Московский Авиационный Институт\\(национальный исследовательский университет)}\\[1.5cm] % Name of your university/college

%----------------------------------------------------------------------------------------
%	LOGO SECTION
%----------------------------------------------------------------------------------------
\includegraphics[width=0.25\textwidth]{mai.png}\\[1cm]
%----------------------------------------------------------------------------------------
\vspace{40px}

%----------------------------------------------------------------------------------------
%	TITLE SECTION
%----------------------------------------------------------------------------------------
\textsc{\Large Курсовая работа по курсу\\ \guillemotleft Численные методы\guillemotright}\\[0.5cm]

\HRule \\[0.4cm]
{ \huge \bfseries Метод Монте-Карло для вычисления кратных интегралов}\\[0.4cm] % Title of your document
\HRule \\[1.5cm]

%----------------------------------------------------------------------------------------
%	AUTHOR SECTION
%----------------------------------------------------------------------------------------
\begin{minipage}{0.4\textwidth}
\begin{flushleft} \large
\emph{Студент:}\\
Макаров Никита, 8о-306Б
\end{flushleft}
\end{minipage}
~
\begin{minipage}{0.4\textwidth}
\begin{flushright} \large
\emph{Руководитель:} \\
Ревизников Д.Л.
\end{flushright}
\end{minipage}\\[2cm]

%----------------------------------------------------------------------------------------
%	DATE SECTION
%----------------------------------------------------------------------------------------

\vspace{120px}
{\large Москва\\2015}\\[2cm] % Date, change the \today to a set date if you want to be precise

\vfill % Fill the rest of the page with whitespace

\end{titlepage}



%----------------------------------------------------------------------------------------
%	СОДЕРЖАНИЕ
%----------------------------------------------------------------------------------------
\tableofcontents

%----------------------------------------------------------------------------------------
%	ВВЕДЕНИЕ
%----------------------------------------------------------------------------------------
\newpage
\section{Введение}


Методами Монте-Карло называют численные методы решения математических задач при помощи моделирования случайных величин. Однако, решать методами Монте-Карло можно любые математические задачи, а не только задачи вероятностного происхождения, связанные со случайными величинами.

Важнейшим приемом построения методов Монте-Карло является сведение задачи к расчету математических ожиданий. Так как математические ожидания чаще всего представляют собой обычные интегралы, то центральное положение в теории метода Монте-Карло занимают методы вычисления интегралов.

Преимущества недетерминированных методов особенно ярко проявляются при решении задач большой размерности, когда применение традиционных детерминированных методов затруднено или совсем невозможно.

Границы между простым и сложным, возможным и невозможным существуют всегда, но с развитием вычислительной техники сдвигаются вдаль. До появления электронных вычислительных машин (ЭВМ) методы Монте-Карло не могли стать универсальными численными методами, ибо моделирование случайных величин вручную - весьма трудоемкий процесс. Развитию методов Монте-Карло способствовало бурное развитие ЭВМ. Алгоритмы Монте-Карло сравнительно легко программируются и позволяют производить расчеты во многих задачах, недоступных для классических численных методов. Так как совершенствование ЭВМ продолжается, есть все основания ожидать дальнейшего развития методов Монте-Карло и дальнейшего расширения области их применения.


%----------------------------------------------------------------------------------------
%	СУЩНОСТЬ
%----------------------------------------------------------------------------------------
\newpage
\section{Идея метода Монте-Карло}

%----------------------------------------------------------------------------------------
%	ПОГРЕШНОСТЬ
%----------------------------------------------------------------------------------------
\newpage
\section{Погрешность метода Монте-Карло}

%----------------------------------------------------------------------------------------
%	УСРЕДНЕНИЕ 
%----------------------------------------------------------------------------------------
\newpage
\section{Вычисление интегралов усреднением подынтегральной функции}

%----------------------------------------------------------------------------------------
%	ТЕСТИРОВАНИЕ 
%----------------------------------------------------------------------------------------
\newpage
\section{Тестирование разработанного ПО}

%----------------------------------------------------------------------------------------
%	КОД 
%----------------------------------------------------------------------------------------
\newpage
\section{Исходный код программы}

\begin{lstlisting}[language=MATLAB]
classdef MonteCarlo
    % Integration by Monte Carlo method
    
    methods(Static)
        
        %% randInRange: Returns random value in range [a,b].
        function [x] = randInRange(a,b)
            n = length(a);
            x = zeros(1,n);
            for i = 1:n
                x(i) = a(i) + (b(i) - a(i)) .* rand();
            end
        end
        
        %% checkPoint: Checks, is point x in G area, or not.
        function [inArea] = checkPoint(x,G)
            n = length(G); % old version: n = length(x) !!!
            inArea = 1;
            for i = 1:n
                inArea = inArea && G{i}(x);
            end
        end
        
        %% ndIntegral: Computing n-dimensional definite integral 
         % at G area which no more than 
         % n-dimensional parallelepiped with properties a and b.
        function [I,c] = ndIntegral(f,a,b,G,N,t_beta)
            %t_beta = 3; % beta = 0.997
            fSum = 0; % sum of computed values f(x)
            fSumSquared = 0; % squared sum of f(x)
            n = 0; % amount of points found in G
            
            % generating N random vectors x
            for i = 1:N
                x = MonteCarlo.randInRange(a,b);
                % check conditions, x must be in [a,b]
                inArea = MonteCarlo.checkPoint(x,G); % bool value
                % adding f(x)
                if (inArea)
                    fSum = fSum + f(x);
                    fSumSquared = fSumSquared + f(x)^2;
                    n = n + 1;
                end
            end
           
            c = inf;
            
            % computing n-dimensional volume of figure
            V = prod(b-a);
            
            % computing integral value
            I = V * fSum / N;
            
            if (nargin == 6)
                fAvg = fSum / n;
                fSquaredAvg = fSumSquared / n;
                
                Omega = n / N;
                
                % computing standard deviation
                S1 = sqrt(fSquaredAvg - fAvg^2);
                S2 = sqrt(Omega * (1 - Omega));
                
                % computing error
                error = V*t_beta*(Omega*S1/sqrt(n) + abs(fAvg)*S2/sqrt(N));
                c = V*t_beta*(sqrt(Omega)*S1 + fAvg*S2);

                disp('Error of integration:');
                disp(error);
            end
        end
        
        %% TESTS
        
        %% test1d: computing 1-dimensional definite integral
        function [I] = test1d(N0, maxError, t_beta)
            f = @(x) 1/sqrt(2*pi) * exp(-(x^2)/2);
            
            % conditions of G area
            x1Cond = @(x) (0<=x(1) && x(1)<=3);
            G = {x1Cond};
            
            % limitations for every element in x
            a(1) = 0; b(1) = 3;
            
            [I,c] = MonteCarlo.ndIntegral(f,a,b,G,N0,t_beta);
            disp('First approximation of integral value:');
            disp(I);
            
            % computing min necessary N
            N = ceil((c/maxError)^2);
            
            if (N > N0)
                disp('Minimal necessary N:');
                disp(N);

                % computing more correct integral value
                [I] = MonteCarlo.ndIntegral(f,a,b,G,N,t_beta);
                disp('Integral value:');
                disp(I);
            end
        end
        
        %% test2d: computing 2-dimensional definite integral
        function [I] = test2d(N0, maxError, t_beta)
            f = @(x) x(1) + x(2);
            
            % conditions of G area
            x1Cond = @(x) (0<=x(1) && x(1)<=2);
            x2Cond = @(x) (x(1)^2<=x(2) && x(2)<=2*x(1));
            G = {x1Cond,x2Cond};
            
            % limitations for every element in x
            a(1) = 0; b(1) = 2;
            a(2) = 0; b(2) = 4;
            
            [I,c] = MonteCarlo.ndIntegral(f,a,b,G,N0,t_beta);
            disp('First approximation of integral value:');
            disp(I);
            
            % computing min necessary N
            N = ceil((c/maxError)^2);
            
            if (N > N0)
                disp('Minimal necessary N:');
                disp(N);

                % computing more correct integral value
                [I] = MonteCarlo.ndIntegral(f,a,b,G,N,t_beta);
                disp('Integral value:');
                disp(I);
            end
        end
        
        %% test3d: computing 3-dimensional definite integral
        function [I] = test3d(N0, maxError, t_beta)
            f = @(x) 10*x(1);
            
            % conditions of G area
            x1Cond = @(x) (0<=x(1) && x(1)<=1);
            x2Cond = @(x) (0<=x(2) && x(2)<=sqrt(1-x(1)^2));
            x3Cond = @(x) (0<=x(3) && x(3)<=((x(1)^2+x(2)^2)/2));
            G = {x1Cond,x2Cond,x3Cond};
            
            % limitations for every element in x
            a(1) = 0; b(1) = 1;
            a(2) = 0; b(2) = 1;
            a(3) = 0; b(3) = 1;
            
            [I,c] = MonteCarlo.ndIntegral(f,a,b,G,N0,t_beta);
            disp('First approximation of integral value:');
            disp(I);
            
            % computing min necessary N
            N = ceil((c/maxError)^2);
            
            if (N > N0)
                disp('Minimal necessary N:');
                disp(N);

                % computing more correct integral value
                [I] = MonteCarlo.ndIntegral(f,a,b,G,N,t_beta);
                disp('Integral value:');
                disp(I);
            end
        end

        %% test6d: computing 6-dimensional definite integral
        function [I] = test6d(N0, maxError, t_beta)
            % Solving the problem of the 
            % mutual attraction of two material bodies.
            
            gravityConst = 6.67e-11; % gravitational constant
            m1 = 6e+24; % mass of the Earth
            m2 = 7.35e+22; % mass of the Moon
            r = 384467000; % distance between Earth and Moon
            p1 = 5520; % avg density of Earth
            p2 = 3346; % avg density of Moon
            R1 = 6367000; % Earth radius
            R2 = 1737000; % Moon radius
            
            dist = @(x) sqrt((x(1)-(r+x(4)))^2 + ...
                             (x(2)-x(5))^2 + ...
                             (x(3)-x(6))^2);
            
            fx = @(x) (x(1)-(r+x(4))) / ((dist(x))^3);
            fy = @(x) (x(2)-x(5)) / ((dist(x))^3);
            fz = @(x) (x(3)-x(6)) / ((dist(x))^3);
            
            % conditions of G area
            x1Cond = @(x) (x(1)^2+x(2)^2+x(3)^2<=R1^2);
            x2Cond = @(x) (x(4)^2+x(5)^2+x(6)^2<=R2^2);
            G = {x1Cond,x2Cond};
            
            % limitations for every element in x
            a(1) = -R1; b(1) = R1;
            a(2) = -R1; b(2) = R1;
            a(3) = -R1; b(3) = R1;
            a(4) = -R2; b(4) = R2;
            a(5) = -R2; b(5) = R2;
            a(6) = -R2; b(6) = R2;
            
            [FxI,c] = MonteCarlo.ndIntegral(fx,a,b,G,N0,t_beta);
            Fx = gravityConst * p1 * p2 * FxI;
            Nx = ceil((c/maxError)^2)
            
            [FyI,c] = MonteCarlo.ndIntegral(fy,a,b,G,N0,t_beta);
            Fy = gravityConst * p1 * p2 * FyI;
            Ny = ceil((c/maxError)^2)
            
            [FzI,c] = MonteCarlo.ndIntegral(fz,a,b,G,N0,t_beta);
            Fz = gravityConst * p1 * p2 * FzI;
            Nz = ceil((c/maxError)^2)
            
            I = sqrt(Fx^2 + Fy^2 + Fz^2);
            disp('Integral value:');
            disp(I);
                        
            F_correct = gravityConst * m1 * m2 / r^2;
            
            diff = abs(I - F_correct);
            disp('Difference with correct answer:');
            disp(diff);
        end
        
    end
    
end
\end{lstlisting}


\end{document}